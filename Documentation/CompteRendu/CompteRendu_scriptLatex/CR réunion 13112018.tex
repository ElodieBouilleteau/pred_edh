\documentclass[11pt]{report}

\usepackage[utf8]{inputenc}
\usepackage[T1]{fontenc}
\usepackage[french]{babel}

\title{Compte-rendu réunion du 13/11/2018}
\author{Élodie Bouilleteau}
\date{Mardi 13 Octobre 2018}

\begin{document}

\maketitle
\renewcommand{\thesection}{\arabic{section}} 
\section{Information générales}
Sujet du PRED : Analyse de la satisfaction client
\\
	\begin{itemize}
		\item Où : Bureau de NN
		\item Quand : 13/10/2018
		\item Participants : EB et NN
	\end{itemize}

\section{Abréviations utilisées}

\subsection{Projet de Recherche et Développement (PRED)}
- Projet de Recherche et Développement (PRED)

\subsection{Encadrants (EN)}
 	\begin{itemize}
		\item Nicolas Normand (NN)
		\item Stéphane Gaudin (SG)
	\end{itemize}
    
\subsection{Etudiants (ET)}
- Élodie Bouilleteau (EB)

\subsection{Encadrants et étudiants (ALL)}
- Encadrants et étudiants (ALL)

\newpage

\section{Point abordés lors de la réunion}
	\subsection{Définition}
	Un réseau de neurone est composer de plusieurs couches qui possèdent chacune un rôle spécifique. Les premières couches servent à trouer les caractéristiques dans l'image puis les dernières couches servent à préciser la classe chercher.\\\\
	La classification d'objet dans une image consiste à détecter à quel type de classe un objet appartient.\\\\
	La segmentation d'une image consiste à savoir où sont situés les objets dans l'image.

	\subsection{Réseau pré-entrainer}
		\begin{itemize}
			\item Les réseaux de neurones pré-entrainer existent déjà et sont performants. Réutiliser ces réseaux peut correspondre à notre problématique et nous fera gagner du temps. \\
			\item il existe 2 méthodes d'utilisation des réseaux de neurones pré-entrainer :
			\begin{enumerate}
				\item Si le réseau donne en sortie une image avec l'emplacement des personnes, alors on pourra réutiliser cette image et en déduire nos indicateurs.
				\item Si le réseau donne des valeurs de classification des objets (ex: cat : 0.25, dog : 0.47...), on est obligé de revenir au couche précédente. On peut spécifier les couches du réseau que l'on garde lors de l'apprentissage et réapprendre les couches finales. C'est la technique du find tuning. \\
			\end{enumerate}
		\end{itemize}

\section{Tâches}
	\begin{itemize}
		\item Lire les articles liés au réseau de neurone pré-entrainer. \\
		\item Comparer les différents réseaux de neurone pré-entrainer trouvé qui correspondent à notre besoin. \\
		\item Connaître les problématiques des méthodes trouvés.\\
		\item Regarder les performances et la faisabilité des méthodes.\\
		\item Présenter un article (bien décrire la problématique de l'article, décrire la structure et ce que l'on peut réutiliser.) via un power point.
	\end{itemize}

\section{Questions}
	\begin{itemize}
		\item En quoi les réseaux de neurone répondent à la problématique ?\\		
		\item Comment ont été entrainer les réseaux de neurone ? (Si il s'agit d'un modèle pré-entrainer disponible au complet, nous pouvons le réutiliser. S'il s'agit d'un modèle issus d'un modèle pré-entrainer mais qui a subit quelques modifications facilement  codable, alors nous pouvons le réutiliser. S'il s'agit d'un modèle dont le code est trop complexe pour être réutiliser, nous ne pouvons pas le récupérer.\\
		\item Sur quelles données ont été entrainer les réseaux ?\\
		\item Le réseau de neurone comprend il le tracking ?
	\end{itemize}
		
\section{Prochaine réunion}
	\begin{itemize}
		\item Où : Bureau de NN
		\item Quand : 20/11/2018 à 16 h 00
		\item Participants : NN, EB
	\end{itemize}

\end{document}