\documentclass[11pt]{report}

\usepackage[utf8]{inputenc}
\usepackage[T1]{fontenc}
\usepackage[french]{babel}

\title{Compte-rendu réunion du 20/11/2018}
\author{Élodie Bouilleteau}
\date{Mardi 20 Novembre 2018}

\begin{document}

\maketitle
\renewcommand{\thesection}{\arabic{section}} 
\section{Information générales}
Sujet du PRED : Analyse de la satisfaction client
\\
	\begin{itemize}
		\item Où : Bureau de NN
		\item Quand : 20/10/2018
		\item Participants : EB et NN
	\end{itemize}

\section{Abréviations utilisées}

\subsection{Projet de Recherche et Développement (PRED)}
- Projet de Recherche et Développement (PRED)

\subsection{Encadrants (EN)}
 	\begin{itemize}
		\item Nicolas Normand (NN)
		\item Stéphane Gaudin (SG)
	\end{itemize}
    
\subsection{Etudiants (ET)}
- Élodie Bouilleteau (EB)

\subsection{Encadrants et étudiants (ALL)}
- Encadrants et étudiants (ALL)

\newpage

\section{Point abordés lors de la réunion}
Présentation d'un article de suivi de personnes mais cet article était trop complexe. Nous avons lus des articles de suivi de personnes et chercher des méthodes de suivi de personnes à intégré dans l'état de l'art.

\section{Tâches}
	\begin{itemize}
		\item Rédaction de l'état de l'art en priorité
	\end{itemize}

%\section{Questions}
	%\begin{itemize}
		%\item En quoi les réseaux de neurone répondent à la problématique ?\\		
		%\item Comment ont été entrainer les réseaux de neurone ? (Si il s'agit d'un modèle pré-entrainer disponible au complet, nous pouvons le réutiliser. S'il s'agit d'un modèle issus d'un modèle pré-entrainer mais qui a subit quelques modifications facilement  codable, alors nous pouvons le réutiliser. S'il s'agit d'un modèle dont le code est trop complexe pour être réutiliser, nous ne pouvons pas le récupérer.\\
		%\item Sur quelles données ont été entrainer les réseaux ?\\
		%\item Le réseau de neurone comprend il le tracking ?
	%\end{itemize}
		
\section{Prochaine réunion}
	\begin{itemize}
		\item Où : Bureau de NN
		\item Quand : 29/11/2018 à 16 h 00
		\item Participants : NN, EB
	\end{itemize}

\end{document}