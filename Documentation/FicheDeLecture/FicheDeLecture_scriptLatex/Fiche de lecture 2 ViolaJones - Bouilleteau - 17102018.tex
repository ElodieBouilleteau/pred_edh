\documentclass[11pt]{report}

\usepackage{hyperref}
\usepackage[utf8]{inputenc}
\usepackage[T1]{fontenc}
\usepackage[french]{babel}

\title{Fiche de Lecture 2}
\author{Élodie Bouilleteau}
\date{Mercredi 17 Octobre 2018}

\begin{document}

\maketitle
\renewcommand{\thesection}{\arabic{section}} 

\section{Référence de l'article}
	\begin{flushleft}
		\textbf{Titre} : Rapid object detection using a boosted cascade of simple features.
	\end{flushleft}
	\textbf{Auteurs} : Paul Viola et Michael Jones.\\\\
	\textbf{Conférence} : Conference on Computer Vision and Pattern Recognition\\\\
	\textbf{Date de parution} : 2001\\\\
	\textbf{Lien} : \url{https://ieeexplore.ieee.org/document/990517}
	
\section{Situation des auteurs}
Paul Viola, ancien professeur au MIT et vice-président des sciences pour
Amazon Air est chercheur en vision par ordinateur. Michael Jones travaillais
en 2001 pour le laboratoire Compaq CRL situé à Cambridge.

\section{Introduction}
Cette publication traite d’un framework robuste de détection d’objet
visuel rapide et possédant un taux de précision élevé mis en place par les
auteurs.

\section{Définition}
\subsection{Image intégrale}
L'image intégrale peut être calculée à partir d'une image en utilisant quelques opérations par pixel. Une fois calculée, chacune de ces caractéristiques peut être calculée à n'importe quelle échelle ou emplacement en temps constant.

\section{Méthode de détection}
Leur méthode se découpe en 3 clés de contributions :
\begin{enumerate}
	\item La première phase est l’introduction d’une nouvelle représentation
d’image appelée «Image intégrale», qui permet de calculer très rapidement
les caractéristiques utilisées par le détecteur.
	\item La deuxième phase est un algorithme d’apprentissage, basé sur Ada-
Boost, qui sélectionne un petit nombre de caractéristiques visuelles
critiques de Haar et produit des classificateurs extrêmement efficaces.
	\item La troisième étape est une méthode pour combiner des classificateurs
dans une "cascade" qui permet d’éliminer rapidement les régions
d’arrière-plan de l’image tout en se concentrant sur les régions
prometteuses.
\end{enumerate}
Deux caractéristiques sont relevé pour détecter les visages : La première
caractéristique mesure la différence d’intensité entre la région des yeux et la
région sur les joues supérieures. La fonctionnalité tire profit de l’observation
que la région des yeux est souvent plus sombre que les joues. La deuxième
caractéristique compare les intensités dans les régions des yeux aux intensités
sur la région du nez.

\section{Performance}
Fonctionnant sur des images de 384 x 288 pixels, les visages sont détectés
à 15 images par seconde sur un Intel Pentium III 700 MHz classique.

\section{Conclusion}
Cette article explique une méthode rapide de détection de visage en sélectionnant
des caractéristiques grâce à une variance de l’algorithme AdaBoost
et en apprenant ses caractéristiques à un classificateur.

\end{document}