\documentclass[12pt]{fiche-rd-info}
\usepackage[utf8]{inputenc}
\usepackage[T1]{fontenc}

\begin{document}

\authorA{Elodie}{Bouilleteau}

\begin{fichesuivi}{8 octobre 2018}{12 octobre 2018}
	\tempstravailA{12}{30}

	\begin{travaileffectue}
		\begin{itemize}
			\item tâche 1 : Compréhension du sujet ; simplicité ; achevée.
			\item tâche 2 : Reformulation du sujet ; difficulté : moyenne ; achevée.
			\item tâche 3 : Recherche générale du sujet sur Wikipédia ; simplicité ; achevée.
			\item tâche 4 : \'Ecriture du contexte, problématique ; simplicité ; achevée.
			\item tâche 5 : Lecture de quelques documents (thèses, articles...) sur le sujet ; difficulté : moyenne ; réalisée à  $30$ \%
		\end{itemize}
	\end{travaileffectue}

	%\begin{travailnoneffectue}
		%\begin{itemize}
			%\item t\^ache 1 : raisons ; reports, annulations ; etc. ;
		%\end{itemize}
	%\end{travailnoneffectue}

	\begin{echange}
		\begin{itemize}
			%\item questions ;
			%\item réponses ;
			%\item éléments de clarification, compréhension ;
			%\item choix, orientations, redéfinitions ;
			%\item etc.
			\item Est ce que l’on garde l’analyse en temps réel ? ; Est ce que la restitution visuelle des résultats est
importante ?
			\item Le temps réel n’est pas une priorité. ; La restitution est utile car elle permet de présenter le résultat
au sein du pôle innovation de U GIE IRIS ;
			\item Les indicateurs sont trop nombreux. Il faut se spécialisé dans un niveau ;
		\end{itemize}
	\end{echange}

	\begin{planification}
		\begin{itemize}
			\item recherches à effectuer ;
			\item articles à lire, comprendre et analyser ;
			\item proposer des indicateurs pertinents ;
		\end{itemize}
	\end{planification}
\end{fichesuivi}

\end{document}
