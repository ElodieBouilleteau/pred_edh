\documentclass[12pt]{fiche-rd-info}
\usepackage[utf8]{inputenc}
\usepackage[T1]{fontenc}

\begin{document}

\authorA{Elodie}{Bouilleteau}

\begin{fichesuivi}{15 octobre 2018}{19 octobre 2018}
	\tempstravailA{10}{30}

	\begin{travaileffectue}
		\begin{itemize}
			\item tâche 1 : Fiche de lecture sur deux documents scientifiques traitant de la méthode Viola-Jones ; difficulté : moyenne ; achevée.
			\item tâche 2 : Précision des problèmes à résoudre ; difficulté : moyenne ; achevée.
			\item tâche 3 : Recherche d'une banque de données de vidéos libre de droits ;  réalisée à  $75$ \% ; achevée.
			%\item tâche 4 : \'Ecriture du contexte, problématique ; simplicité ; achevée.
			%\item tâche 5 : Lecture de quelques documents (thèses, articles...) sur le sujet ; difficulté : moyenne ; réalisée à  $30$ \%
		\end{itemize}
	\end{travaileffectue}

	%\begin{travailnoneffectue}
		%\begin{itemize}
			%\item t\^ache 1 : raisons ; reports, annulations ; etc. ;
		%\end{itemize}
	%\end{travailnoneffectue}

	\begin{echange}
		\begin{itemize}
			%\item questions ;
			%\item réponses ;
			%\item éléments de clarification, compréhension ;
			%\item choix, orientations, redéfinitions ;
			%\item etc.
			\item Précision des problèmes du sujet avec Nicolas NORMAND ;
				\begin{enumerate}
					\item Méthode de détection de visage.
					\item Méthode d'identification de 2 personnes identiques d'une image à l'autre.
					\item Méthode de traçage de personne relié à la méthode d'identification.
					\item Sous-problème : la gestion des occlusions.
				\end{enumerate}
			\item Recherche générale sur toutes les méthodes possibles et performantes lié à notre sujet ;
			%\item etc.
		\end{itemize}
	\end{echange}

	\begin{planification}
		\begin{itemize}
			%\item recherches à effectuer ;
			%\item articles à lire, comprendre et analyser ;
			%\item codes à développer ;
			%\item etc.
			\item Recherche sur les méthodes de détection, de traçage et d'identification récente et performante à effectuer ;
			\item articles à lire, comprendre et analyser ;
			%\item proposer des indicateurs pertinents ;
		\end{itemize}
	\end{planification}
\end{fichesuivi}

\end{document}
