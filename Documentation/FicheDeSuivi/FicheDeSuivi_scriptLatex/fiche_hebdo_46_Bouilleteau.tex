\documentclass[12pt]{fiche-rd-info}
\usepackage[utf8]{inputenc}
\usepackage[T1]{fontenc}

\begin{document}

\authorA{Elodie}{Bouilleteau}

\begin{fichesuivi}{12 novembre 2018}{18 novembre 2018}
	\tempstravailA{11}{00}

	\begin{travaileffectue}
		\begin{itemize}
			\item tâche 1 : Fiche de lecture sur la méthode de détection YOLO ; difficulté : moyenne ; achevée.
			\item tâche 2 : Fiche de lecture 2 méthode de détection de personnes en temps réel  ; difficulté : moyenne ; achevée.
			\item tâche 3 : Recherche d'une banque de données de vidéos libres de droits ; achevée.
			\item tâche 4 : Recherche d'article sur le tracking d'objet avec code source ; simplicité ; achevée.
			%\item tâche 5 : Lecture de quelques documents (thèses, articles...) sur le sujet ; difficulté : moyenne ; réalisée à  $30$ \%
		\end{itemize}
	\end{travaileffectue}

	\begin{travailnoneffectue}
		\begin{itemize}
				\item tâche 1 : Rédaction de l'état de l'art, recherche bibliographique non terminée ;
			%\item tâche 1 : raisons ; reports, annulations ; etc. ;
		\end{itemize}
	\end{travailnoneffectue}

	\begin{echange}
		\begin{itemize}
			\item Concentration sur des méthodes avec le code source open source et réutilisable ;
			%\item questions ;
			%\item réponses ;
			%\item éléments de clarification, compréhension ;
			%\item choix, orientations, redéfinitions ;
			%\item etc.
			%\item etc.
		\end{itemize}
	\end{echange}

	\begin{planification}
		\begin{itemize}
			%\item recherches à effectuer ;
			%\item articles à lire, comprendre et analyser ;
			%\item codes à développer ;
			%\item etc.
			\item Lire les articles de tracking de personne et en faire des fiches de lecture (3 max);
			\item Faire une présentation power-point d'un article ;
			\item Rédaction de l'état de l'art ;
			\item Faire le diagramme de Gantt ;
			\item Faire un Excel synthétique de toutes les solutions possibles avec leurs critères de sélection ;
		\end{itemize}
	\end{planification}
\end{fichesuivi}

\end{document}
