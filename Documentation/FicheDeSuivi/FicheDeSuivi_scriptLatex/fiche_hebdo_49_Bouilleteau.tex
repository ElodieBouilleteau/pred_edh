\documentclass[12pt]{fiche-rd-info}
\usepackage[utf8]{inputenc}
\usepackage[T1]{fontenc}

\begin{document}

\authorA{Elodie}{Bouilleteau}

\begin{fichesuivi}{3 décembre 2018}{9 décembre 2018}
	\tempstravailA{10}{15}

	\begin{travaileffectue}
		\begin{itemize}
			\item tâche 1 : Diaporama de soutenance ; difficulté : simple ; achevée.
			\item tâche 2 : Correction du rapport ; difficulté : moyenne ; réalisée à  $45$ \%.
			\item tâche 3 : Rédaction feuille indicateur pour faire la vidéo de test ; difficulté : moyenne ; achevée.
			%\item tâche 5 : Lecture de quelques documents (thèses, articles...) sur le sujet ; difficulté : moyenne ; réalisée à  $30$ \%
		\end{itemize}
	\end{travaileffectue}

	%\begin{travailnoneffectue}
		%\begin{itemize}
			%	\item tâche 1 : Présentation power-point , reporter au lundi de la semaine prochaine ;
			%\item tâche 1 : raisons ; reports, annulations ; etc. ;
		%\end{itemize}
	%\end{travailnoneffectue}

	%\begin{echange}
		%\begin{itemize}
			%\item Explication des différentes approches sur le suivi de personnes ;
			%\item questions ;
			%\item réponses ;
			%\item éléments de clarification, compréhension ;
			%\item choix, orientations, redéfinitions ;
			%\item etc.
			%\item etc.
		%\end{itemize}
	%\end{echange}

	\begin{planification}
		\begin{itemize}
			%\item recherches à effectuer ;
			%\item articles à lire, comprendre et analyser ;
			%\item codes à développer ;
			%\item etc.
			\item Implémenter la proposition 1 et 2;
		\end{itemize}
	\end{planification}
\end{fichesuivi}

\end{document}
